%%%%%%%%%%%%%%%%%%%%%%%%%%%%%%%%%%%%%%%%%%%%%%%%%%%%%%%%%%%%%%%%%%%%%%%%%%%%%%%
% CSCE 645 - Geometric Modeling: Final Project Report
% High-Fidelity Mesh Smoothing for Medical Brain MRI Data
% Texas A&M University
%%%%%%%%%%%%%%%%%%%%%%%%%%%%%%%%%%%%%%%%%%%%%%%%%%%%%%%%%%%%%%%%%%%%%%%%%%%%%%%

\documentclass[11pt,letterpaper]{article}

% ============================================================================
% PACKAGES
% ============================================================================
\usepackage[utf8]{inputenc}
\usepackage[T1]{fontenc}
\usepackage{times}                    % Times New Roman font
\usepackage[margin=1in]{geometry}
\usepackage{graphicx}
\usepackage{amsmath,amssymb,amsfonts}
\usepackage{algorithm}
\usepackage{algpseudocode}
\usepackage{booktabs}
\usepackage{multirow}
\usepackage{caption}
\usepackage{subcaption}
\usepackage{hyperref}
\usepackage{cleveref}
\usepackage{xcolor}
\usepackage{float}
\usepackage{enumitem}
\usepackage{fancyhdr}
\usepackage{lastpage}
\usepackage{setspace}
\usepackage{parskip}

% ============================================================================
% CONFIGURATION
% ============================================================================
\definecolor{tamumaroon}{RGB}{80,0,0}
\hypersetup{
    colorlinks=true,
    linkcolor=tamumaroon,
    citecolor=tamumaroon,
    urlcolor=tamumaroon
}

% Header/Footer
\pagestyle{fancy}
\fancyhf{}
\fancyhead[L]{CSCE 645: Geometric Modeling}
\fancyhead[R]{Final Project Report}
\fancyfoot[C]{Page \thepage\ of \pageref{LastPage}}
\renewcommand{\headrulewidth}{0.4pt}
\renewcommand{\footrulewidth}{0.4pt}
\setlength{\headheight}{14pt}

% Line spacing
\onehalfspacing

% Graphics paths (generated artifacts)
\graphicspath{{../outputs/figures/final_report/}{../outputs/figures/mesh_comparisons/}}

% ============================================================================
% TITLE
% ============================================================================

% Manual title strings (avoid issues with escaped control sequences)
\newcommand{\ReportTitle}{High-Fidelity Feature-Preserving Mesh Smoothing for Brain Tumor Segmentation}
\newcommand{\ReportSubtitle}{Comparative Evaluation on 20 BraTS Cases}

\author{Shubham Vikas Mhaske \\ \textit{Department of Computer Science and Engineering} \\ \textit{Texas A\&M University} \\ \texttt{shubhammhaske@tamu.edu} \\ NetID: 334003509}

\date{CSCE 645: Geometric Modeling -- Fall 2025 \\ Instructor: Professor John Keyser \\ \today}

% ============================================================================
% DOCUMENT
% ============================================================================
\begin{document}

\begin{center}
{\LARGE\textbf{\ReportTitle}}\\[0.35em]
{\large \ReportSubtitle}\\[0.85em]
{\normalsize Shubham Vikas Mhaske}\\
{\textit{Department of Computer Science and Engineering, Texas A\&M University}}\\
	exttt{shubhammhaske@tamu.edu}\\
NetID: 334003509\\[0.75em]
{\normalsize CSCE 645: Geometric Modeling -- Fall 2025\\ Instructor: Professor John Keyser\\ \today}
\end{center}

	hispagestyle{fancy}

% ----------------------------------------------------------------------------
% ABSTRACT
% ----------------------------------------------------------------------------
\begin{abstract}
This project presents a mesh smoothing pipeline for medical brain MRI segmentation data from the BraTS dataset. Starting from voxelized 3D tumor masks, we extract surface meshes via Marching Cubes and apply smoothing to remove staircase artifacts while preserving clinically relevant geometry.

We evaluate five methods on \textbf{20 BraTS cases} (5 BraTS-GLI + 15 BraTS2021): two classical baselines (Laplacian, Taubin $\lambda|\mu$) and three feature-preserving methods developed in this project (Geodesic Heat Diffusion, Information-Theoretic Smoothing, and Anisotropic Tensor Smoothing). Metrics include \textbf{volume change} (\%), \textbf{smoothness reduction} (\%), \textbf{triangle aspect-ratio improvement} (\%), and \textbf{runtime} (ms).

Across 20 cases, Laplacian and Geodesic Heat achieve the strongest smoothing (\textbf{97.45\%} and \textbf{97.01\%} smoothness reduction) but introduce shrinkage (\textbf{-0.92\%} and \textbf{-0.82\%} mean volume change). Taubin and Information-Theoretic provide near volume neutrality (\textbf{+0.056\%} and \textbf{+0.042\%}) with strong smoothing (\textbf{88.96\%} and \textbf{84.35\%}) and fast runtimes (\textbf{24.6ms} and \textbf{44.4ms}). Anisotropic Tensor best preserves volume (\textbf{-0.022\%}) and minimizes distortion (lowest AR change) but is slower (\textbf{126.0ms} mean, up to \textbf{328.8ms}) and less aggressive in smoothing (\textbf{59.52\%}).
\end{abstract}

	extbf{Keywords:} Mesh smoothing, volume preservation, medical imaging, brain tumor segmentation, BraTS, surface reconstruction

% ----------------------------------------------------------------------------
% 1. INTRODUCTION
% ----------------------------------------------------------------------------
\section{Introduction}
\label{sec:introduction}

\subsection{Problem Statement}

Medical imaging workflows increasingly rely on 3D visualization of anatomical structures extracted from volumetric data such as MRI and CT scans. Brain tumor visualization is particularly critical for neurosurgical planning, where accurate 3D models help surgeons understand tumor location, extent, and relationship to critical brain structures. When segmentation masks---binary or multi-label volumes indicating tissue types---are converted to surface meshes using algorithms like Marching Cubes~\cite{lorensen1987marching}, the resulting surfaces exhibit significant staircase artifacts due to discrete voxel resolution (typically 1mm isotropic for BraTS data).

These artifacts manifest as:
\begin{itemize}[noitemsep]
    \item \textbf{High-frequency noise:} Jagged edges along voxel boundaries creating non-smooth surfaces
    \item \textbf{Poor triangle quality:} Elongated and degenerate triangles with aspect ratios far from ideal
    \item \textbf{Curvature artifacts:} Artificial high-curvature regions at voxel corners
    \item \textbf{Visual noise:} Rendering artifacts that obscure fine anatomical details
\end{itemize}

The fundamental challenge in mesh smoothing for medical applications is balancing competing objectives:
\begin{enumerate}[noitemsep]
    \item \textbf{Artifact Removal:} Eliminating staircase effects from voxelized surfaces
    \item \textbf{Volume Preservation:} Maintaining accurate volumetric measurements for clinical use (tumor size monitoring)
    \item \textbf{Feature Preservation:} Retaining anatomically significant boundaries between tumor core, edema, and healthy tissue
    \item \textbf{Curvature Fidelity:} Preserving natural curvature characteristics for realistic visualization
    \item \textbf{Computational Efficiency:} Enabling real-time interactive exploration in clinical settings
\end{enumerate}

\subsection{Contributions}

This project makes the following key contributions:
\begin{enumerate}[noitemsep]
    \item \textbf{Algorithm Suite:} Implementation and comparative evaluation of five smoothing methods: Laplacian, Taubin $\lambda|\mu$, and three feature-preserving approaches (Geodesic Heat, Information-Theoretic, Anisotropic Tensor)
    \item \textbf{Evaluation on 20 cases:} End-to-end benchmarking on 20 BraTS meshes (5 BraTS-GLI + 15 BraTS2021) with consistent metrics (volume change, smoothness reduction, aspect ratio change, runtime)
    \item \textbf{Efficient implementation:} Vectorized sparse-matrix implementations enabling per-mesh runtimes in the tens of milliseconds for most methods
    \item \textbf{3D visualization artifacts:} Publication-ready figures and before/after mesh snapshots with consistent lighting and camera settings
    \item \textbf{Interactive application:} Streamlit-based interface for loading NIfTI masks, selecting algorithms/parameters, and exporting smoothed meshes
\end{enumerate}

% ----------------------------------------------------------------------------
% 2. BACKGROUND
% ----------------------------------------------------------------------------
\section{Background and Related Work}
\label{sec:background}

\subsection{Mesh Smoothing Fundamentals}

Mesh smoothing is a core operation in geometric modeling that adjusts vertex positions to reduce surface noise while preserving overall shape. The most common approach is \textit{Laplacian smoothing}, which iteratively moves each vertex toward the centroid of its neighbors:

\begin{equation}
    \mathbf{v}_i' = \mathbf{v}_i + \lambda \cdot L(\mathbf{v}_i)
    \label{eq:laplacian}
\end{equation}

where $L(\mathbf{v}_i) = \frac{1}{|N(i)|} \sum_{j \in N(i)} \mathbf{v}_j - \mathbf{v}_i$ is the discrete Laplacian operator, $N(i)$ denotes the one-ring neighborhood of vertex $i$, and $\lambda \in (0, 1)$ controls the smoothing strength.

While effective at removing high-frequency noise, Laplacian smoothing suffers from \textit{shrinkage}---the mesh contracts toward its center of mass with each iteration. This is problematic for medical applications where accurate volume measurements are clinically significant.

\subsection{Taubin Smoothing}

Taubin~\cite{taubin1995signal} proposed viewing mesh smoothing through the lens of signal processing. The key insight is that Laplacian smoothing acts as a low-pass filter that attenuates high-frequency geometric features (noise) but also affects low frequencies (overall shape), causing shrinkage.

The solution is a two-step process that alternates between shrinking ($\lambda > 0$) and expanding ($\mu < 0$) steps:

\begin{align}
    \mathbf{v}' &= \mathbf{v} + \lambda \cdot L(\mathbf{v}) \label{eq:taubin1} \\
    \mathbf{v}'' &= \mathbf{v}' + \mu \cdot L(\mathbf{v}') \label{eq:taubin2}
\end{align}

where $0 < \lambda < -\mu$. Taubin showed that with appropriate parameter selection (typically $\lambda = 0.5$, $\mu = -0.53$), this process effectively removes high-frequency noise while preserving low-frequency shape, resulting in near-zero volume change.

\subsection{Curvature Analysis}

Differential geometry provides tools for analyzing surface properties through curvature measures. For discrete meshes, we employ:

\textbf{Mean Curvature} ($H$) is computed using the cotangent Laplacian~\cite{meyer2003discrete}:
\begin{equation}
    H(\mathbf{v}_i) = \frac{1}{4A} \sum_{j \in N(i)} (\cot \alpha_{ij} + \cot \beta_{ij})(\mathbf{v}_i - \mathbf{v}_j)
    \label{eq:mean_curvature}
\end{equation}
where $\alpha_{ij}$ and $\beta_{ij}$ are the angles opposite to edge $(i,j)$ in the two adjacent triangles, and $A$ is the mixed Voronoi area.

\textbf{Gaussian Curvature} ($K$) is computed using the angle defect formula:
\begin{equation}
    K(\mathbf{v}_i) = \frac{2\pi - \sum_j \theta_j}{A_{\text{mixed}}}
    \label{eq:gaussian_curvature}
\end{equation}
where $\theta_j$ are the interior angles at vertex $i$ in incident triangles.

\subsection{BraTS Dataset}

The Brain Tumor Segmentation (BraTS) Challenge~\cite{baid2021rsna} provides multi-institutional MRI scans with expert annotations of glioma sub-regions. In this project we evaluate meshes extracted from 20 cases spanning two naming conventions present in our data: 5 ``BraTS-GLI'' cases and 15 ``BraTS2021'' cases.

The segmentation masks include:
\begin{itemize}[noitemsep]
    \item \textbf{Label 1:} Necrotic tumor core (NCR) -- central necrotic regions
    \item \textbf{Label 2:} Peritumoral edematous tissue (ED) -- surrounding edema
    \item \textbf{Label 3:} GD-enhancing tumor (ET) -- actively growing tumor tissue
\end{itemize}

In this project, we focus on Label 1 (necrotic tumor core) as it presents a challenging smoothing scenario with complex geometry and clinical significance for volumetric assessment.

The 20 evaluated cases include:
\begin{itemize}[noitemsep]
    \item \textbf{BraTS-GLI:} BraTS-GLI-00001-000, BraTS-GLI-00001-001, BraTS-GLI-00013-000, BraTS-GLI-00013-001, BraTS-GLI-00015-000
    \item \textbf{BraTS2021:} BraTS2021\_00000, \_00002, \_00003, \_00005, \_00006, \_00008, \_00009, \_00011, \_00012, \_00014, \_00016, \_00017, \_00018, \_00019, \_00020
\end{itemize}

\begin{table}[H]
\centering
\caption{Mesh complexity statistics for the 20 evaluated BraTS cases}
\label{tab:samples}
\begin{tabular}{@{}lrrr@{}}
	oprule
	extbf{Statistic} & \textbf{Vertices} & \textbf{Faces} & \textbf{Volume (mm$^3$)} \\
\midrule
Minimum & 5,980 & 11,952 & 18,426 \\
Mean & 45,876 & 91,747 & 488,596 \\
Maximum & 118,766 & 237,528 & 1,639,347 \\
\bottomrule
\end{tabular}
\end{table}

% ----------------------------------------------------------------------------
% 3. METHODOLOGY
% ----------------------------------------------------------------------------
\section{Methodology}
\label{sec:methodology}

\subsection{Pipeline Overview}

Our processing pipeline consists of five stages, as illustrated in Figure~\ref{fig:pipeline}:

\begin{figure}[H]
    \centering
    \includegraphics[width=0.98\textwidth]{fig6_pipeline.png}
    \caption{High-fidelity mesh processing pipeline. The six-stage pipeline transforms NIfTI brain tumor segmentation masks into smooth, volume-preserving surface meshes. Each stage is annotated with its function: NIfTI loading, volume coarsening with label mapping, Marching Cubes surface extraction, Taubin $\lambda|\mu$ smoothing, quality metrics computation, and final output with visualization.}
    \label{fig:pipeline}
\end{figure}

\begin{enumerate}
    \item \textbf{NIfTI Loading:} Load 3D segmentation masks from NIfTI format with affine transformation
    \item \textbf{Volume Processing:} Coarsen label volume to reduce memory requirements while preserving boundaries
    \item \textbf{Marching Cubes:} Extract isosurface at threshold 0.5 using scikit-image implementation
    \item \textbf{Mesh Smoothing:} Apply selected algorithm (Laplacian, Taubin, Geodesic Heat, Info-Theoretic, or Anisotropic)
    \item \textbf{Quality Metrics:} Compute volume change, smoothness reduction, aspect ratio change, and runtime
\end{enumerate}

\subsection{Implemented Algorithms}

\subsubsection{Laplacian Smoothing}
Standard iterative vertex averaging as described in Equation~\ref{eq:laplacian}, with $\lambda = 0.5$.

\subsubsection{Taubin \texorpdfstring{$\lambda|\mu$}{lambda|mu} Smoothing}
Two-step volume-preserving smoothing as described in Equations~\ref{eq:taubin1}-\ref{eq:taubin2}, with $\lambda = 0.5$ and $\mu = -0.53$.

\subsubsection{Geodesic Heat Diffusion (Ours)}
We implement a feature-preserving smoothing strategy inspired by heat diffusion on surfaces. Intuitively, the heat kernel defines an intrinsic notion of locality on the surface; vertex updates are guided by diffusion while respecting geodesic structure.

\subsubsection{Information-Theoretic Smoothing (Ours)}
This method adapts smoothing strength based on information content in local geometric neighborhoods, reducing smoothing in regions that appear structurally informative while still removing voxel staircase artifacts.

\subsubsection{Anisotropic Tensor Smoothing (Ours)}
Anisotropic smoothing applies direction-dependent filtering using a local tensor field, aiming to preserve sharp geometric features while smoothing along directions consistent with the surface.

\subsection{Quality Metrics}

\begin{table}[H]
\centering
\caption{Quality metrics used for evaluation}
\label{tab:metrics}
\begin{tabular}{@{}lll@{}}
\toprule
\textbf{Metric} & \textbf{Description} & \textbf{Target} \\
\midrule
Volume Change (\%) & Relative change in enclosed volume (signed) & $|\Delta V|$ small \\
Smoothness Reduction (\%) & Reduction of curvature/noise proxy & Maximize \\
AR Change (\%) & Change in mean triangle aspect ratio & Maximize \\
Runtime (ms) & Wall-clock time per mesh & Minimize \\
\bottomrule
\end{tabular}
\end{table}

% ----------------------------------------------------------------------------
% 4. IMPLEMENTATION
% ----------------------------------------------------------------------------
\section{Implementation}
\label{sec:implementation}

\subsection{Technology Stack}

The pipeline is implemented in Python 3.9+ using the following libraries:
\begin{itemize}[noitemsep]
    \item \textbf{NumPy/SciPy:} Vectorized operations and sparse matrix representation
    \item \textbf{PyVista:} 3D mesh processing and visualization
    \item \textbf{NiBabel:} NIfTI file I/O
    \item \textbf{scikit-image:} Marching Cubes surface extraction
    \item \textbf{Streamlit:} Interactive web application
\end{itemize}

\subsection{Vectorized Implementation}

All algorithms are implemented using vectorized NumPy operations for efficiency. The adjacency structure is built once using sparse matrices and reused across iterations:

\begin{algorithm}[H]
\caption{Vectorized Taubin Smoothing}
\label{alg:taubin}
\begin{algorithmic}[1]
\Require Vertices $V \in \mathbb{R}^{n \times 3}$, Faces $F \in \mathbb{Z}^{m \times 3}$, iterations $k$
\Ensure Smoothed vertices $V'$
\State Build sparse adjacency matrix $A$ from $F$
\State Compute degree vector $d = A \cdot \mathbf{1}$
\For{$i = 1$ to $k$}
    \State $C \gets A \cdot V \oslash d$ \Comment{Neighbor centroid (vectorized)}
    \State $V \gets V + \lambda (C - V)$ \Comment{Shrink step}
    \State $C \gets A \cdot V \oslash d$
    \State $V \gets V + \mu (C - V)$ \Comment{Expand step}
\EndFor
\State \Return $V$
\end{algorithmic}
\end{algorithm}

\subsection{Interactive Application}

The Streamlit application provides:
\begin{itemize}[noitemsep]
    \item File upload for NIfTI segmentation masks
    \item Algorithm selection with parameter controls
    \item Real-time 3D mesh visualization
    \item Quality metrics dashboard
    \item Mesh export functionality (OBJ, STL, PLY)
\end{itemize}

% ----------------------------------------------------------------------------
% 5. EXPERIMENTAL RESULTS
% ----------------------------------------------------------------------------
\section{Experimental Results}
\label{sec:results}

\subsection{Dataset and Setup}

Experiments were conducted on \textbf{20 BraTS cases} (Table~\ref{tab:samples}). For each case, we extract a surface mesh from the segmentation mask and apply five smoothing methods. Unless otherwise noted, all metrics are reported as averages across the 20 cases.

\subsection{Quantitative Results (n=20)}

Table~\ref{tab:results_summary} summarizes the aggregate performance.

\begin{table}[H]
\centering
\caption{Aggregate results across 20 BraTS cases (mean $\pm$ std). Runtime also shows min--max in parentheses.}
\label{tab:results_summary}
\begin{tabular}{@{}lrrrr@{}}
	oprule
	extbf{Algorithm} & \textbf{$\Delta$Volume (\%)} & \textbf{Smoothness Red. (\%)} & \textbf{AR Change (\%)} & \textbf{Time (ms)} \\
\midrule
Laplacian & $-0.92 \pm 0.79$ & $97.45 \pm 0.78$ & $58.03 \pm 108.72$ & $17.21$ (2.61--43.60) \\
Taubin & $+0.06 \pm 0.05$ & $88.96 \pm 1.93$ & $13.47 \pm 0.34$ & $24.57$ (4.20--62.91) \\
	extit{Geodesic Heat (Ours)} & $-0.82 \pm 0.71$ & $97.01 \pm 0.91$ & $53.41 \pm 100.38$ & $27.06$ (4.20--69.84) \\
	extit{Info-Theoretic (Ours)} & $+0.04 \pm 0.04$ & $84.35 \pm 2.24$ & $12.24 \pm 0.28$ & $44.40$ (6.46--114.95) \\
	extit{Anisotropic (Ours)} & $-0.02 \pm 0.02$ & $59.52 \pm 1.84$ & $10.06 \pm 0.22$ & $126.05$ (17.12--328.81) \\
\bottomrule
\end{tabular}
\end{table}

	extbf{Key trends:}
\begin{itemize}[noitemsep]
    \item \textbf{Smoothing strength:} Laplacian and Geodesic Heat achieve the highest smoothness reduction ($\approx 97\%$) but also the highest shrinkage.
    \item \textbf{Volume neutrality:} Taubin and Info-Theoretic are near volume-neutral (mean $|\Delta V| < 0.1\%$) while still producing strong smoothing.
    \item \textbf{Runtime:} Laplacian/Taubin/Geodesic run in tens of milliseconds; Info-Theoretic is typically sub-100ms; Anisotropic is slower and has the highest variance.
\end{itemize}

Figure~\ref{fig:comparison} visualizes the aggregate comparison across metrics:

\begin{figure}[H]
    \centering
    \includegraphics[width=0.95\textwidth]{fig1_algorithm_comparison.png}
    \caption{Aggregate comparison across 20 BraTS cases. Metrics summarize signed volume change (\%), smoothness reduction (\%), aspect ratio change (\%), and runtime (ms).}
    \label{fig:comparison}
\end{figure}

\subsection{Trade-offs and Pareto Structure}

\begin{figure}[H]
    \centering
    \includegraphics[width=0.95\textwidth]{fig2_tradeoff_scatter.png}
    \caption{Trade-off scatter plot across 20 cases, illustrating the relationship between smoothing strength, volume change, and runtime.}
    \label{fig:tradeoff}
\end{figure}

\begin{figure}[H]
    \centering
    \includegraphics[width=0.95\textwidth]{fig3_radar_comparison.png}
    \caption{Radar summary of normalized metrics across all five algorithms (higher is better after normalization).}
    \label{fig:radar}
\end{figure}

\begin{figure}[H]
    \centering
    \includegraphics[width=0.95\textwidth]{fig4_processing_time.png}
    \caption{Runtime distribution across 20 cases. Most methods run in the tens of milliseconds; Anisotropic is slower and more variable.}
    \label{fig:runtime}
\end{figure}

\begin{figure}[H]
    \centering
    \includegraphics[width=0.95\textwidth]{fig5_novelty_diagram.png}
    \caption{Positioning of the three proposed methods relative to classical baselines, highlighting differing trade-offs between feature preservation, smoothing strength, and compute cost.}
    \label{fig:novelty}
\end{figure}

\subsection{Qualitative Mesh Comparisons}
Figure~\ref{fig:mesh_compare} shows representative before/after mesh visualizations on a BraTS case.

\begin{figure}[H]
    \centering
    \begin{subfigure}{0.49\textwidth}
        \centering
        \includegraphics[width=\textwidth]{BraTS-GLI-00013-001_taubin_comparison_close.png}
        \caption{Original vs Taubin}
    \end{subfigure}
    \hfill
    \begin{subfigure}{0.49\textwidth}
        \centering
        \includegraphics[width=\textwidth]{BraTS-GLI-00013-001_all_algorithms_close.png}
        \caption{Original + all algorithms}
    \end{subfigure}
    \caption{Close-up mesh snapshots with consistent camera/lighting. These visuals complement the quantitative metrics by showing how each method removes voxel staircase artifacts while affecting volume and fine-scale geometry.}
    \label{fig:mesh_compare}
\end{figure}

\begin{figure}[H]
    \centering
    \includegraphics[width=0.98\textwidth]{fig7_summary_table.png}
    \caption{Summary table figure aggregating the main metrics across all algorithms.}
    \label{fig:summary_table}
\end{figure}

% ----------------------------------------------------------------------------
% 6. DISCUSSION
% ----------------------------------------------------------------------------
\section{Discussion}
\label{sec:discussion}

\subsection{Volume Preservation vs. Smoothing Strength}
The results highlight the classic tension between aggressive smoothing and shape preservation. Laplacian and Geodesic Heat remove the most high-frequency artifacts but also induce the largest negative volume change. Taubin and Info-Theoretic occupy a strong middle ground, providing substantial smoothing with near-zero mean volume change.

\subsection{Runtime Considerations}
Across the 20 evaluated cases, Laplacian, Taubin, and Geodesic Heat typically run in tens of milliseconds per mesh. Info-Theoretic is also commonly sub-100ms. Anisotropic is the slowest method and shows higher variance, reaching 300ms+ on the largest meshes.

\subsubsection{Implementation Architecture}

The vectorized implementation using NumPy and SciPy sparse matrices enables efficient CPU-only processing:
\begin{itemize}[noitemsep]
    \item \textbf{Sparse Matrix Representation:} Adjacency and Laplacian matrices use CSR format, reducing memory from $O(n^2)$ to $O(|E|)$
    \item \textbf{Vectorized Operations:} All vertex updates computed in parallel via matrix multiplication
    \item \textbf{Optimized Curvature:} Cotangent Laplacian and angle defect computed using broadcasting, avoiding explicit loops
    \item \textbf{Memory Efficiency:} 118K vertex mesh requires ~45MB RAM (vertices + faces + sparse matrices)
\end{itemize}

\textbf{GPU Acceleration Potential:} Novel algorithms could achieve 10-50x speedup with CUDA implementation of heat kernel computation and entropy calculations, enabling near-real-time performance.

\subsection{Limitations and Considerations}

\begin{enumerate}[noitemsep]
    \item \textbf{Manifold Assumption:} The current implementation assumes manifold meshes without boundaries. Non-manifold edges or isolated vertices would require preprocessing.
    \item \textbf{High Curvature Regions:} Extremely high curvature regions (curvature $> 2$ standard deviations) may experience slight feature degradation. This could be addressed with curvature-adaptive iteration counts.
    \item \textbf{Label Resolution:} Sub-voxel boundaries cannot be recovered from a discrete segmentation mask; smoothing can only operate on the extracted mesh.
    \item \textbf{Parameter Sensitivity:} Some methods expose parameters that trade smoothing strength against preservation. A principled, data-driven selection strategy is a promising direction.
    \item \textbf{Memory Scaling:} For meshes exceeding 500K vertices, the sparse matrix operations may require chunked processing or out-of-core algorithms.
    \item \textbf{Multi-Label Boundaries:} Current implementation handles two-label boundaries; extension to three-way junctions (where three labels meet) requires special treatment.
\end{enumerate}

% ----------------------------------------------------------------------------
% 7. CONCLUSION
% ----------------------------------------------------------------------------
\section{Conclusion}
\label{sec:conclusion}

This project developed and evaluated a mesh smoothing pipeline for brain tumor segmentation meshes on \textbf{20 BraTS cases}. We compared two classical baselines (Laplacian and Taubin) with three feature-preserving methods developed in this project (Geodesic Heat, Information-Theoretic, and Anisotropic Tensor). All claims and figures in this report are based on measured runs and generated artifacts.

\subsection{Practical Recommendations}
\begin{itemize}[noitemsep]
    \item \textbf{Fast smoothing with strong artifact removal:} Laplacian (fastest) or Geodesic Heat (similar smoothness, slightly slower), noting both introduce measurable shrinkage.
    \item \textbf{Balanced default choice:} Taubin provides strong smoothing with near-zero mean volume change and low runtime.
    \item \textbf{Volume-first smoothing:} Information-Theoretic offers near volume neutrality with moderate smoothing.
    \item \textbf{Most conservative updates:} Anisotropic Tensor minimizes distortion and volume change but is slower and less aggressive in smoothing.
\end{itemize}

\subsection{Future Work}

\begin{itemize}[noitemsep]
    \item GPU acceleration using CUDA for larger meshes
    \item Adaptive iteration count based on convergence criteria
    \item Integration with deep learning-based mesh processing
    \item Extension to multi-material mesh generation
    \item Clinical validation study with radiologists
\end{itemize}

% ----------------------------------------------------------------------------
% REFERENCES
% ----------------------------------------------------------------------------
\bibliographystyle{plain}
\begin{thebibliography}{10}

\bibitem{baid2021rsna}
Baid, U., et al.
\newblock The RSNA-ASNR-MICCAI BraTS 2021 Benchmark on Brain Tumor Segmentation and Radiogenomic Classification.
\newblock \textit{arXiv preprint arXiv:2107.02314}, 2021.

\bibitem{desbrun1999implicit}
Desbrun, M., Meyer, M., Schr{\"o}der, P., and Barr, A.H.
\newblock Implicit Fairing of Irregular Meshes using Diffusion and Curvature Flow.
\newblock In \textit{SIGGRAPH '99}, pages 317--324, 1999.

\bibitem{fleishman2003bilateral}
Fleishman, S., Drori, I., and Cohen-Or, D.
\newblock Bilateral Mesh Denoising.
\newblock \textit{ACM Transactions on Graphics}, 22(3):950--953, 2003.

\bibitem{garland1997surface}
Garland, M. and Heckbert, P.S.
\newblock Surface Simplification Using Quadric Error Metrics.
\newblock In \textit{SIGGRAPH '97}, pages 209--216, 1997.

\bibitem{lorensen1987marching}
Lorensen, W.E. and Cline, H.E.
\newblock Marching Cubes: A High Resolution 3D Surface Construction Algorithm.
\newblock In \textit{SIGGRAPH '87}, pages 163--169, 1987.

\bibitem{meyer2003discrete}
Meyer, M., Desbrun, M., Schr{\"o}der, P., and Barr, A.H.
\newblock Discrete Differential-Geometry Operators for Triangulated 2-Manifolds.
\newblock In \textit{Visualization and Mathematics III}, pages 35--57. Springer, 2003.

\bibitem{nealen2006laplacian}
Nealen, A., Igarashi, T., Sorkine, O., and Alexa, M.
\newblock Laplacian Mesh Optimization.
\newblock In \textit{Proceedings of ACM GRAPHITE}, pages 381--389, 2006.

\bibitem{taubin1995signal}
Taubin, G.
\newblock A Signal Processing Approach to Fair Surface Design.
\newblock In \textit{SIGGRAPH '95}, pages 351--358, 1995.

\end{thebibliography}

\end{document}
